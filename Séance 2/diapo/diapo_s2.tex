\documentclass{beamer}
\usepackage[utf8]{inputenc}
\usepackage[T1]{fontenc}
\usepackage{lmodern}
\usepackage{amsmath}
\usepackage{amssymb}
\usepackage[french]{babel}
% \usepackage[cache=false]{minted}
\usepackage{pythontex}

\usepackage{pgf,tikz}
\usepackage{mathrsfs}
\usepackage{tkz-graph}

%load tikz libraries
\usepackage{tkz-euclide}
\usepackage{listings}
\usepackage{graphicx}
\usetikzlibrary{arrows}
\usetikzlibrary{calc,patterns,shapes.geometric}


\usetheme{Boadilla}
\usecolortheme{beaver}

%------------------------------------------------------------
%This block of code defines the information to appear in the
%Title page
\title %optional
{L Codent L Créent}

\subtitle{Séance 2}

\author[ENSTA Bretagne] % (optional)
{C. \textsc{Hinard} \and M. \textsc{Miallier} \and T. \textsc{Prévost}}

\institute[] % (optional)
{
  ENSTA Bretagne
  \and
  Collège Croas ar Pennoc
}

\date[Séance 2] % (optional)
{10 mai 2022}

%\logo{\includegraphics[height=1cm]{overleaf-logo}}

%End of title page configuration block
%------------------------------------------------------------



%------------------------------------------------------------
%The next block of commands puts the table of contents at the
%beginning of each section and highlights the current section:

\AtBeginSection[]
{
  \begin{frame}
    \frametitle{Plan}
    \tableofcontents[currentsection]
  \end{frame}
}

\AtBeginSubsection[]
{
   \begin{frame}
      \frametitle{Plan}
      \tableofcontents[currentsection, currentsubsection]
   \end{frame}
}
%------------------------------------------------------------


\begin{document}

%The next statement creates the title page.
\frame{\titlepage}


%---------------------------------------------------------
%The next statement creates the table of contents.
\begin{frame}
\frametitle{Plan}
\tableofcontents
\end{frame}
%---------------------------------------------------------


\section{Rappels}\label{sec:rappels}
   \subsection{Opérations}\label{subsec:operations}

%---------------------------------------------------------
%Changing visibility of the text
\begin{frame}
\frametitle{Les bases}

\begin{itemize}
    \item<1-> \texttt{5 + 2} $\Longrightarrow$ \pause \texttt{7} \pause
    \item<2-> \texttt{5 - 2} $\Longrightarrow$ \pause \texttt{3} \pause
    \item<3-> \texttt{5 * 2} $\Longrightarrow$ \pause \texttt{10} \pause
    \item<4-> \texttt{5 / 2} $\Longrightarrow$ \pause \texttt{2.5}

\end{itemize}
\end{frame}

\begin{frame}
\frametitle{Mais aussi\dots}

\begin{itemize}%python arithmetics
   \item<1-> \texttt{5 \% 2} $\Longrightarrow$ \pause \texttt{7} \pause
   \item<2-> \texttt{5 ** 2} $\Longrightarrow$ \pause \texttt{25} \pause
   \item<3-> \texttt{5 // 2} $\Longrightarrow$ \pause \texttt{2} \pause
   \item<4-> \texttt{5 < 2} $\Longrightarrow$ \pause \texttt{True} \pause
   \item<5-> \texttt{5 > 2} $\Longrightarrow$ \pause \texttt{False} \pause
   \item<6-> \texttt{5 <= 2} $\Longrightarrow$ \pause \texttt{True} \pause
   \item<7-> \texttt{5 >= 2} $\Longrightarrow$ \pause \texttt{False} \pause
   \item<8-> \texttt{5 == 2} $\Longrightarrow$ \pause \texttt{False} \pause
   \item<9-> \texttt{5 != 2} $\Longrightarrow$ \pause \texttt{True} \pause
   \item<10-> \texttt{5==5 and 2==3} $\Longrightarrow$ \pause \texttt{False} \pause
   \item<11-> \texttt{5==5 or 2==3} $\Longrightarrow$ \pause \texttt{True}
\end{itemize}
\end{frame}

   \subsection{\texttt{If, then, else}}\label{subsec:if-then-else}

%---------------------------------------------------------


\section{Et aujourd'hui, on fait quoi ?}\label{sec:et-aujourdhui-on-fait-quoi}
   \subsection{Boucle \texttt{for}}\label{subsec:boucle-for}
\begin{frame}[fragile]
   \frametitle{Somme des entiers de 1 à 20}
   \begin{pyverbatim}
            somme = 0
            for i in range(1, 21):
               somme = somme + i
            print(somme)
   \end{pyverbatim}
\end{frame}

\begin{frame}[fragile]
   \frametitle{Table de 5}
   \begin{pyverbatim}
            for i in range(11):
               print("5 * " + i + " = " + 5*i)
   \end{pyverbatim}
\end{frame}

   \subsection{Boucle \texttt{while}}\label{subsec:boucle-while}

\begin{frame}[fragile]
   \frametitle{La somme inconnue}

   \begin{pyverbatim}
         somme = 0
         n = 0
         while somme < 1000:
            n = n + 1
            somme = somme + n
         print(n)
   \end{pyverbatim}
\end{frame}

\begin{frame}[fragile]
   \frametitle{L'avion}

   \begin{pyverbatim}
         vitesse = 180
         t = 0
         while vitesse < 340:
            vitesse += 10
            t += 1
         print(t)
   \end{pyverbatim}
\end{frame}

\section{Mais ensuite ?}\label{sec:mais-ensuite}
   \subsection{Fonctions}\label{subsec:fonctions}

\begin{frame}[fragile]
   \frametitle{Un peu de maths ?}

   \begin{center}
      \begin{tikzpicture}
         \draw (0,-1) rectangle (4,1) node[pos=.5] {$f$};
         \draw[stealth-] (0,0) -- (-2,0) node[left] {$x$};
         \draw[-stealth] (4,0) -- ++(2,0) node[right] {$f(x) = \cdots$};
      \end{tikzpicture}
   \end{center}

   \begin{pyverbatim}
      def f(x):
         return ...
   \end{pyverbatim}
\end{frame}

\begin{frame}[fragile]
   \frametitle{Encore une somme !}

   \begin{pyverbatim}
         def somme(n):
            s = 0
            for i in range(1, n+1):
               s += i
            return s
   \end{pyverbatim}
\end{frame}

\begin{frame}[fragile]
   \frametitle{Le double}

   \begin{pyverbatim}
         def double(x):
            return 2 * x
         x = double(3)
         print(x)
   \end{pyverbatim}
\end{frame}

\begin{frame}[fragile]
   \frametitle{Liste des pairs}

   \begin{pyverbatim}
         def nombres_pairs(liste):
            liste_pairs = []
            for nombre in liste:
               if nombre % 2 == 0:
                  liste_pairs.append(nombre)
            return liste_pairs
   \end{pyverbatim}
\end{frame}

\begin{frame}[fragile]
   \frametitle{Première et dernière lettre}

   \begin{pyverbatim}
      def premiere_derniere_lettre(mot):
         return mot[0] + "..." + mot[-1]
   \end{pyverbatim}
\end{frame}

\section{On dessine !}\label{sec:on-dessine}
   \subsection{Bonjour \texttt{turtle} !}\label{subsec:bonjour-turtle}
\begin{frame}
   \frametitle{Le module \texttt{turtle}}

   \texttt{import turtle as t}

   \begin{center}
      \includegraphics[scale=.6]{img/franklin}
   \end{center}

\end{frame}

\begin{frame}[fragile]
   \frametitle{Création du fichier}

   \begin{pyverbatim}
      import turtle as t

      t.shape("turtle")
      t.exitonclick()
   \end{pyverbatim}

   \begin{block}{Remarque}
      Sans l'instruction \texttt{import turtle as t}, Python ne sait pas que l'on veut utiliser le module \texttt{turtle}, donc ça ne marchera pas !
   \end{block}
\end{frame}

   \subsection{À vous !}\label{subsec:a-vous}

\begin{frame}[fragile]
   \frametitle{Carré}

   \begin{pyverbatim}
            def carre(x):
               t.pendown()
               for i in range(4):
                  t.forward(x)
                  t.left(90)
               t.penup()
   \end{pyverbatim}
\end{frame}

\begin{frame}[fragile]
   \frametitle{La maison}

   \begin{pyverbatim}
      def maison():
   \end{pyverbatim}
   \pause
   \begin{pyverbatim}
         # les murs
         carre(80)
   \end{pyverbatim}
   \pause
   \begin{pyverbatim}
         # on va à l'emplacement de la porte
         t.forward(25)
   \end{pyverbatim}
   \pause
   \begin{pyverbatim}
         # on dessine la porte
         carre(30)
   \end{pyverbatim}
   \pause
   \begin{pyverbatim}
         # on va à la position du début du toit
         t.forward(25)
         t.left(90)
         t.forward(80)
   \end{pyverbatim}
\end{frame}

\begin{frame}[fragile]
   \frametitle{Maison (suite et fin)}

   \begin{pyverbatim}
         # on dessine le toit (triangle équilatéral)
         t.pendown()
   \end{pyverbatim}
   \pause
   \begin{pyverbatim}
         t.left(30)
         t.forward(80)
   \end{pyverbatim}
   \pause
   \begin{pyverbatim}
         t.left(120)
         t.forward(80)
   \end{pyverbatim}
   \pause
   \begin{pyverbatim}
         t.penup()
   \end{pyverbatim}
   \pause
   \begin{pyverbatim}
         t.home()
   \end{pyverbatim}
\end{frame}

\begin{frame}[fragile]
   \frametitle{Le village}

   \begin{pyverbatim}
      def village(nb_maisons):
         x, y = 0, 0
         t.home()
   \end{pyverbatim}
   \pause
   \begin{pyverbatim}
         for n in range(nb_maisons):
   \end{pyverbatim}
   \pause\begin{pyverbatim}
            maison()
   \end{pyverbatim}
   \pause
   \begin{pyverbatim}
            x += 90
   \end{pyverbatim}
   \pause
   \begin{pyverbatim}
            t.setpos(x, 0)
   \end{pyverbatim}
   \pause
   \begin{pyverbatim}

      village(3)
   \end{pyverbatim}
\end{frame}

   \subsection{Pour finir}\label{subsec:pour-finir}

\begin{frame}
   \frametitle{Petit dessin pour finir}

   \begin{pyverbatim}
      t.penup()
      t.setpos(0,0)
      for i in range(0, 361, 5):
         t.pendown()
         t.circle(80)
         t.penup()
         t.circle(200, i)
   \end{pyverbatim}
\end{frame}
%---------------------------------------------------------
%Highlighting text
%\begin{frame}
%\frametitle{Sample frame title}
%
%In this slide, some important text will be
%\alert{highlighted} because it's important.
%Please, don't abuse it.
%
%\begin{block}{Remark}
%Sample text
%\end{block}
%
%\begin{alertblock}{Important theorem}
%Sample text in red box
%\end{alertblock}
%
%\begin{examples}
%Sample text in green box. The title of the block is ``Examples".
%\end{examples}
%\end{frame}
%---------------------------------------------------------
\end{document}