\documentclass{beamer}
\usepackage[utf8]{inputenc}
\usepackage[T1]{fontenc}
\usepackage{lmodern}
\usepackage{amsmath}
\usepackage{amssymb}
\usepackage[french]{babel}

\usetheme{Boadilla}
\usecolortheme{beaver}

%------------------------------------------------------------
%This block of code defines the information to appear in the
%Title page
\title %optional
{L Codent L Créent}

\subtitle{Séance 2}

\author[ENSTA Bretagne] % (optional)
{C. \textsc{Hinard} \and M. \textsc{Miallier} \and T. \textsc{Prévost}}

\institute[] % (optional)
{
  ENSTA Bretagne
  \and
  Collège Croas ar Pennoc
}

\date[Séance 2] % (optional)
{10 mai 2022}

%\logo{\includegraphics[height=1cm]{overleaf-logo}}

%End of title page configuration block
%------------------------------------------------------------



%------------------------------------------------------------
%The next block of commands puts the table of contents at the
%beginning of each section and highlights the current section:

\AtBeginSection[]
{
  \begin{frame}
    \frametitle{Plan}
    \tableofcontents[currentsection]
  \end{frame}
}
%------------------------------------------------------------


\begin{document}

%The next statement creates the title page.
\frame{\titlepage}


%---------------------------------------------------------
%The next statement creates the table of contents.
\begin{frame}
\frametitle{Plan}
\tableofcontents
\end{frame}
%---------------------------------------------------------


\section{Rappels}\label{sec:rappels}
   \subsection{Opérations}\label{subsec:operations}

%---------------------------------------------------------
%Changing visibility of the text
\begin{frame}
\frametitle{Les bases}

\begin{itemize}
    \item<1-> \texttt{5 + 2} $\Longrightarrow$ \pause \texttt{7} \pause
    \item<2-> \texttt{5 - 2} $\Longrightarrow$ \pause \texttt{3} \pause
    \item<3-> \texttt{5 * 2} $\Longrightarrow$ \pause \texttt{10} \pause
    \item<4-> \texttt{5 / 2} $\Longrightarrow$ \pause \texttt{2.5}

\end{itemize}
\end{frame}

\begin{frame}
\frametitle{Mais aussi\dots}

\begin{itemize}%python arithmetics
   \item<1-> \texttt{5 \% 2} $\Longrightarrow$ \pause \texttt{7} \pause
   \item<2-> \texttt{5 ** 2} $\Longrightarrow$ \pause \texttt{25} \pause
   \item<3-> \texttt{5 // 2} $\Longrightarrow$ \pause \texttt{2} \pause
   \item<4-> \texttt{5 < 2} $\Longrightarrow$ \pause \texttt{True} \pause
   \item<5-> \texttt{5 > 2} $\Longrightarrow$ \pause \texttt{False} \pause
   \item<6-> \texttt{5 <= 2} $\Longrightarrow$ \pause \texttt{True} \pause
   \item<7-> \texttt{5 >= 2} $\Longrightarrow$ \pause \texttt{False} \pause
   \item<8-> \texttt{5 == 2} $\Longrightarrow$ \pause \texttt{False} \pause
   \item<9-> \texttt{5 != 2} $\Longrightarrow$ \pause \texttt{True} \pause
\end{itemize}
\end{frame}

%---------------------------------------------------------


%---------------------------------------------------------
%Example of the \pause command
\begin{frame}
In this slide \pause

the text will be partially visible \pause

And finally everything will be there
\end{frame}
%---------------------------------------------------------

\section{Second section}

%---------------------------------------------------------
%Highlighting text
\begin{frame}
\frametitle{Sample frame title}

In this slide, some important text will be
\alert{highlighted} because it's important.
Please, don't abuse it.

\begin{block}{Remark}
Sample text
\end{block}

\begin{alertblock}{Important theorem}
Sample text in red box
\end{alertblock}

\begin{examples}
Sample text in green box. The title of the block is ``Examples".
\end{examples}
\end{frame}
%---------------------------------------------------------


%---------------------------------------------------------
%Two columns
\begin{frame}
\frametitle{Two-column slide}

\begin{columns}

\column{0.5\textwidth}
This is a text in first column.
$$E=mc^2$$
\begin{itemize}
\item First item
\item Second item
\end{itemize}

\column{0.5\textwidth}
This text will be in the second column
and on a second tought this is a nice looking
layout in some cases.
\end{columns}
\end{frame}
%---------------------------------------------------------


\end{document}