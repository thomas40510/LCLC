\documentclass[12pt,a4paper, oneside]{article}
\usepackage[utf8]{inputenc}
\usepackage{amsmath}
\usepackage{amsfonts}
\usepackage{amssymb}
\usepackage{graphicx}
\usepackage{pgfplots}
\usepackage{indentfirst}
\usepackage{eurosym}
\usepackage{multicol}
\usepackage[french]{babel}
\usepackage[T1]{fontenc}
\usepackage{lmodern}
\usepackage{array}
\usepackage{stmaryrd}


\usepackage[notitlepage,, fancysections, pagenumber]{sauveterre}
\pagestyle{empty}

\usepackage{titlesec}
\titleformat*{\section}{\centering\Large\bfseries}
\titleformat*{\subsection}{\large\bfseries}
\titleformat*{\subsubsection}{\centering\normalsize\bfseries}
%\titlespacing*{\subsubsection}{0pt}{0pt}{\baselineskip}
%\titleformat*{\paragraph}{\bfseries\itshape}

\renewcommand{\thesection}{--- \Roman{section} ---}

\renewcommand{\thesubsection}{\arabic{section}.\arabic{subsection}.}
\renewcommand{\thesubsubsection}{Partie \Roman{subsubsection}}

\usepackage{pgf,tikz}
\usepackage{mathrsfs}
\usepackage{tkz-graph}

%load tikz libraries
\usepackage{tkz-euclide}
\usepackage{listings}
\usetikzlibrary{arrows}
\usetikzlibrary{calc,patterns,shapes.geometric}


%change sec and subsec numbering
%\renewcommand{\thesection}{\Roman{section}.}
%\renewcommand{\thesubsection}{\arabic{subsection}.}

\renewcommand{\labelenumi}{\textbf{\arabic{enumi}.}}
\renewcommand{\labelenumii}{\textbf{\alph{enumii})}}
\newcommand{\N}{\mathbb{N}}
\newcommand{\R}{\mathbb{R}}
\newcommand{\Rb}{\overline{\mathbb{R}}}
\newcommand{\C}{\mathbb{C}}
\newcommand{\M}{\mathcal{M}}
\newcommand{\epsi}{\varepsilon}
\renewcommand{\parallel}{//}
\newcommand{\Cf}{\mathcal{C}_f}

\renewcommand{\P}{\mathbb{P}}

\newcommand{\MnR}{\M_n(\R)}

\newcommand{\ch}{\operatorname{ch}}
\newcommand{\sh}{\operatorname{sh}}

\renewcommand{\Re}[1]{\operatorname{Re}\left(#1\right)}
\renewcommand{\Im}[1]{\operatorname{Im}\left(#1\right)}

\newcommand{\zb}{\overline{z}}
\newcommand{\conj}[1]{\overline{#1}}
\renewcommand{\mod}[1]{\lvert #1 \rvert}
\newcommand{\scal}{\cdot}
\newcommand{\abs}[1]{\mod{#1}}

\renewcommand{\sup}[2]{\underset{#2}{\operatorname{sup}}~#1}


\newcommand{\mat}[1]{\begin{pmatrix}#1\end{pmatrix}}

% command for vectors / choice btw overrightarrow and vec
\makeatletter
\newcommand{\ve}{\@ifstar{\@veca}{\@vecb}}
\newcommand{\@veca}[1]{\vec{#1}}
\newcommand{\@vecb}[1]{\overrightarrow{#1}}
\makeatother

\newcommand{\Nr}[1]{\lVert #1 \rVert}
\newcommand{\Nrsup}[1]{\lVert #1 \rVert_{\infty}}


\newcommand{\Oij}{\left(O~;~\vec{i},~\vec{j}\right)}
\newcommand{\Oijk}{\left(O~;~\vec{i},~\vec{j},~\vec{k}\right)}
\newcommand{\Ouv}{\left(O~;~\vec{u},~\vec{v}\right)}

\newcommand{\cm}{\, \mathrm{cm}}
\newcommand{\g}{\, \mathrm{g}}
\newcommand{\eur}{\, \text{\euro}}
\newcommand{\gpL}{\, \mathrm{g}\cdot \mathrm{L}^{-1}}

\newcommand{\staritem}{\stepcounter{enumi} \item[$\star$ \labelenumi]}
\newcommand{\stariitem}{\stepcounter{enumii} \item[$\star$ \labelenumii]}

\newcolumntype{M}[1]{>{\centering\arraybackslash}m{#1}}

\renewcommand{\polylogohori}{img/ENSTA-LogoH-NOIR}
\renewcommand{\polylogovert}{img/nologo}
\renewcommand{\polyarmes}{img/nologo}

\usepackage{xcolor}
\definecolor{bleu-sauveterre}{RGB}{237,107,97} %bleu ENSTA

\usepackage{pythontex}

\title[LCLC -- Séance 1]{\og L Codent L Créent \fg{}}
\subtitle{Séance 1}
\author{C. Hinard, M. Miallier, \\T. Prévost}
\date{Collège Croas ar Pennoc,\\ \today}

\begin{document}

\maketitle

\section*{Prise en main}\label{sec:prise-en-main}
  Entrer chaque ligne dans la console Python.
  Observer le résultat.
  \begin{pyverbatim}

    print(2 + 3)
    print(6 - 2)
    print(2 * 6)

    n = 2
    m = 42
    print(n, m)
    print(n * m)

    print(m / n)
    print(m // n)
    print(m % n)

    print(n == m)
    print(n != m)
    print(n == 42)
    print(n == '42')
    s = '42'
    print(m == s)
    print(n > 400)
    print(n <= 430)
    y = n + n
    print(y)
    z = s + s
    print(s)
    # entrée
    i = input("Comment vous appelez-vous ?")
    print(i)

    L = [1, '2', 3]
    print(L)
    print(L[1])
    print(L[0])
  \end{pyverbatim}

  \section{If, then, else}\label{sec:if-then-else}
  On commence à bien prendre en main ce langage merveilleux qu'est Python, on peut commencer à écrire des vrais scripts !
  \subsection{Majeur / Mineur}\label{subsec:majeur-/-mineur}
    Écrire un script qui demande son âge à l'utilisateur, et lui dit s'il est majeur ou mineur.

  \subsection{Nombres pairs}\label{subsec:nombres-pairs}
    Écrire un script Python qui, pour un nombre \texttt{n} choisi, dit s'il est pair ou impair.

  \section{Boucles \texttt{for}}\label{sec:bouclesfor}
    \subsection{Somme des entiers}\label{subsec:somme-des-entiers}
    Écrire un script Python qui donne la somme des entiers de 1 à 20.

    Comment obtenir cette somme plus simplement ? \\ (\textit{indice : utiliser une boucle}.)

    \subsection{Tables de multiplication}\label{subsec:tables-de-multiplication}
      Écrire un script Python qui détaille la table de 3.

      Comment le programmer plus simplement ?

  \section{Boucles \texttt{while}}\label{sec:boucles-while}
    \subsection{Quotient d'une division}\label{subsec:quotient-d'une-division}
      Écrire un script Python qui, à l'aide d'une boucle \texttt{while}, calcule pour deux nombres \texttt{a} et \texttt{b} le quotient de la division de \texttt{a} par \texttt{b}.

    \subsection{Avion}\label{subsec:avion}
      Un avion, initialement à 11\, 000 mètres d'altitude, descend de 300 mètres par minute.
    Au bout de combien de temps passera-t-il sous les 2000 mètres d'altitude ?

    $\Longrightarrow$ Écrire un script permettant d'avoir la réponse.

\section*{--- BOSS FINAL --- \\ Conjecture de Syracuse}
  \textbf{Les règles :}
  \begin{itemize}
    \item On choisit un nombre entier supérieur à 0 ;
    \item s'il est pair, on le divise par 2 ;
    \item s'il est impair, on le multiplie par 3 et on ajoute 1 ;
    \item on recommence jusqu'à obtenir 1.
  \end{itemize}

  $\Longrightarrow$ Programmer cet algorithme sur Python, et le faire tourner avec différents nombres de départ.

\section*{Le boss final mais le vrai cette fois qu'on avait pas vu venir juste après le premier boss final}
  \textbf{Le problème :} On dispose d'un code à 4 chiffres.
  Écrire un script Python permettant de le trouver (en essayant toutes les combinaisons possibles jusqu'à trouver la bonne).

  \textbf{\textit{Pour les matheuses :}} combien de combinaisons différentes vont être testées dans le pire des cas ?

\end{document}
